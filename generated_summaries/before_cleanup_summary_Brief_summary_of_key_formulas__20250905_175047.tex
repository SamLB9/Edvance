\section{FOCUS AREA: Brief summary of key formulas}

\section{Definitions}
\begin{itemize}
\item Conditional probability: $P(B\mid A)$ is the probability of $B$ given $A$ has occurred.
\item Prior probability: Initial probability of a hypothesis before new evidence, written $P(A)$.
\item Posterior probability: Revised probability after observing evidence, written $P(A\mid B)$.
\item Likelihood: Probability of the evidence under a hypothesis, written $P(B\mid A)$.
\item Evidence (marginal probability): Total probability of observed evidence, written $P(B)$.
\end{itemize}

\section{Core formulas}
\begin{itemize}
\item Conditional probability:
$$P(B\mid A)=\frac{P(A\text{ and }B)}{P(A)}.$$
\item Bayes' theorem (two events):
$$P(A\mid B)=\frac{P(B\mid A)\,P(A)}{P(B)}.$$
\item Law of total probability (for evidence):
$$P(B)=\sum_i P(B\mid A_i)\,P(A_i).$$
\item Bayes' theorem (generalized):
$$P(A_j\mid B)=\frac{P(B\mid A_j)\,P(A_j)}{\sum_i P(B\mid A_i)\,P(A_i)}.$$
\end{itemize}

\section{Step-by-step procedure to apply Bayes' theorem}
\begin{enumerate}
\item Identify the hypotheses $A_i$ that partition the sample space.
\item Record priors $P(A_i)$ for each hypothesis.
\item Obtain likelihoods $P(B\mid A_i)$ for the observed evidence $B$.
\item Compute the evidence
$$P(B)=\sum_i P(B\mid A_i)\,P(A_i).$$
\item Compute posteriors
$$P(A_j\mid B)=\frac{P(B\mid A_j)\,P(A_j)}{P(B)}.$$
\end{enumerate}

\section{Intuitive (frequency) method}
\begin{itemize}
\item Assume a convenient total $N$ (e.g., $1000$ or $10000$).
\item For each hypothesis $A_i$ compute $\text{count}_i = N\cdot P(A_i)$.
\item For each hypothesis compute $\text{evidence\_count}_i = \text{count}_i\cdot P(B\mid A_i)$.
\item Total evidence count: $\text{total\_evidence\_count}=\sum_i \text{evidence\_count}_i$.
\item Then
$$P(A_j\mid B)=\frac{\text{evidence\_count}_j}{\text{total\_evidence\_count}}.$$
\item This method is algebraically identical to Bayes' theorem and helps reduce algebra errors.
\end{itemize}

\section{Worked examples (concise)}

\subsection{Orange County cigar smoker}
\begin{itemize}
\item Given: $P(\text{male})=0.51$, $P(\text{female})=0.49$.
\item Likelihoods: $P(\text{smokes}\mid \text{male})=0.095$, $P(\text{smokes}\mid \text{female})=0.017$.
\end{itemize}
Use Bayes:
$$P(\text{male}\mid \text{smokes})=\frac{0.095\cdot 0.51}{0.095\cdot 0.51 + 0.017\cdot 0.49}.$$
Calculation:
\[
\text{numerator}=0.095\times 0.51=0.04845,
\]
\[
\text{denominator}=0.04845 + 0.017\times 0.49=0.04845 + 0.00833 = 0.05678.
\]
Result:
\[
P(\text{male}\mid \text{smokes})\approx\frac{0.04845}{0.05678}\approx 0.853\ (85.3\%).
\]

\subsection{ELT manufacturers (A, B, C) and defect $D$}
\begin{itemize}
\item Given: $P(A)=0.80$, $P(B)=0.15$, $P(C)=0.05$.
\item Likelihoods: $P(D\mid A)=0.04$, $P(D\mid B)=0.06$, $P(D\mid C)=0.09$.
\end{itemize}
Use Bayes for $P(A\mid D)$:
$$P(A\mid D)=\frac{0.80\cdot 0.04}{0.80\cdot 0.04 + 0.15\cdot 0.06 + 0.05\cdot 0.09}.$$
Calculation:
\[
\text{numerator}=0.80\times 0.04=0.032,
\]
\[
\text{denominator}=0.032 + 0.15\times 0.06 + 0.05\times 0.09=0.032 + 0.009 + 0.0045 = 0.0455.
\]
Result:
\[
P(A\mid D)\approx\frac{0.032}{0.0455}\approx 0.703\ (70.3\%).
\]
Frequency check ($N=10000$): $A=8000$ defects$=320$; $B=1500$ defects$=90$; $C=500$ defects$=45$; $P(A\mid D)=320/455\approx 0.703$.

\section{Important relationships and connections}
\begin{itemize}
\item Bayes inverts conditional probabilities: $P(A\mid B)$ relates to $P(B\mid A)$.
\item The denominator in Bayes ensures normalization over all hypotheses.
\item Bayes relies on exhaustive, mutually exclusive hypotheses $A_i$ for a correct denominator.
\item Repeated updating: posteriors can serve as new priors when new evidence arrives.
\item The frequency interpretation often improves intuition and reduces substitution errors.
\end{itemize}

\section{Common pitfalls and quick checks}
\begin{itemize}
\item Do not confuse $P(A\mid B)$ with $P(B\mid A)$.
\item Ensure hypotheses $A_i$ sum to $1$ (exhaustive).
\item Include all relevant $A_i$ in the law of total probability.
\item Check results using the frequency method to validate algebraic results.
\end{itemize}

\section{Concise reference list of formulas (plain text)}
\begin{itemize}
\item $P(B\mid A)=\dfrac{P(A\text{ and }B)}{P(A)}$
\item $P(A\text{ and }B)=P(B\mid A)\,P(A)=P(A\mid B)\,P(B)$
\item $P(A\mid B)=\dfrac{P(B\mid A)\,P(A)}{P(B)}$
\item $P(B)=\sum_i P(B\mid A_i)\,P(A_i)$
\item $P(A_j\mid B)=\dfrac{P(B\mid A_j)\,P(A_j)}{\sum_i P(B\mid A_i)\,P(A_i)}$
\end{itemize}