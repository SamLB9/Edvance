\documentclass[11pt,a4paper]{article}

% --- Encoding and font ---
\usepackage[T1]{fontenc}
\usepackage[utf8]{inputenc}
\usepackage{lmodern}
\usepackage{microtype}

% --- Page geometry ---
\usepackage[a4paper,margin=25mm]{geometry}

% --- Colors and graphics ---
\usepackage{xcolor}
\definecolor{primary}{HTML}{0B6FA4}   % professional blue
\definecolor{secondary}{HTML}{3A6B8A} % darker blue
\definecolor{accent}{HTML}{6E7B8C}    % gray-blue
\definecolor{lightgray}{HTML}{F3F5F7}
\usepackage{graphicx}
\usepackage{float}

% --- Math ---
\usepackage{amsmath,amssymb,amsthm}
\usepackage{mathtools}

% --- Tables and rules ---
\usepackage{booktabs}
\usepackage{array}

% --- Fancy boxes and framing ---
\usepackage{tcolorbox}
\tcbuselibrary{skins,breakable}

% --- Section title styling ---
\usepackage{sectsty}
\allsectionsfont{\color{primary}\sffamily}
\usepackage{titlesec}
\titleformat{\section}{\Large\bfseries\sffamily\color{primary}}{\thesection}{1em}{}
\titleformat{\subsection}{\large\bfseries\sffamily\color{secondary}}{\thesubsection}{0.8em}{}
\titleformat{\subsubsection}{\normalsize\bfseries\sffamily\color{accent}}{\thesubsubsection}{0.6em}{}

% --- Header and footer ---
\usepackage{fancyhdr}
\pagestyle{fancy}
\fancyhf{}
\renewcommand{\headrulewidth}{0.6pt}
\renewcommand{\footrulewidth}{0.4pt}
\renewcommand{\headrule}{\color{lightgray}\hrule height \headrulewidth \vspace{-\headrulewidth}}
\fancyhead[L]{\sffamily\small Course Notes Summary}
\fancyhead[C]{\sffamily\small Focus: Brief summary of main formulas}
\fancyhead[R]{\sffamily\small Source: BayesTheorem.pdf}
\fancyfoot[L]{\sffamily\small Generated: 2025-09-04 13:24:02}
\fancyfoot[C]{}
\fancyfoot[R]{\sffamily\small \thepage}

% --- Hyperlinks ---
\usepackage{hyperref}
\hypersetup{
  colorlinks=true,
  linkcolor=primary,
  urlcolor=secondary,
  citecolor=primary,
  pdftitle={Course Notes Summary - Bayes},
  pdfauthor={Generated},
  pdfproducer={LaTeX}
}

% --- Misc ---
\setcounter{secnumdepth}{3}
\setcounter{tocdepth}{2}
\raggedbottom

% --- Title page custom ---
\usepackage{titling}
\pretitle{\begin{center}\vspace*{1cm}\sffamily}
\posttitle{\par\end{center}\vskip 0.2em}
\preauthor{}\postauthor{}
\predate{}\postdate{}

\begin{document}

% -------------------
% Title page
% -------------------
\begin{titlepage}
  \thispagestyle{empty}
  \begin{center}
    \vspace*{1cm}
    {\sffamily\Huge\bfseries\color{primary} Course Notes Summary \par}
    \vspace{0.8em}
    {\sffamily\Large\color{secondary} Focus Area: Brief summary of main formulas \par}
    \vspace{1.5em}
    \begin{tcolorbox}[
      enhanced,
      width=0.75\textwidth,
      colback=lightgray,
      colframe=primary,
      boxrule=0.8pt,
      sharp corners,
      left=6pt,
      right=6pt,
      top=6pt,
      bottom=6pt
    ]\vspace{0.5em}
      \sffamily
      \begin{tabular}{@{}p{0.28\textwidth} p{0.62\textwidth}@{}}
        \textbf{Source: } & BayesTheorem.pdf \\
        \textbf{Generated:} & 2025-09-04 13:24:02 \\
        \textbf{Summary Type:} & Comprehensive \\
      \end{tabular}
    \end{tcolorbox}

    \vspace{1.5em}
    {\small\sffamily Prepared for quick revision and reference\par}
    \vfill

    % optional: small branded area (logo placeholder)
    % If you have business branding (logo), uncomment and replace "logo.pdf".
    % \includegraphics[height=18mm]{logo.pdf}\\[6pt]
    {\small\sffamily \color{accent} Use this sheet as a step-by-step guide when solving Bayes problems.}
    \vspace{1.8cm}
  \end{center}
\end{titlepage}

\clearpage
\begin{tcolorbox}[colback=blue!5!white,colframe=primary,boxrule=0.8pt,sharp corners,breakable]\vspace{0.5em}
\sffamily
This document condenses the essential definitions, identities, formulas, and practical methods for applying Bayes' Theorem. It is intended as a compact, publication-quality reference for coursework and exam preparation.
\end{tcolorbox}

\section{Key definitions}
\begin{itemize}\setlength{\itemsep}{0.5em}
  \item \textbf{Conditional probability:} $P(B\mid A)$ = probability that $B$ occurs given $A$ has occurred.
  \item \textbf{Joint (intersection) probability:} $P(A\ \text{and}\ B)\equiv P(A\cap B)$.
  \item \textbf{Prior probability:} initial probability of an event before new information, e.g., $P(A)$.
  \item \textbf{Posterior probability:} revised probability after new information, e.g., $P(A\mid B)$.
  \item \textbf{Likelihood:} $P(B\mid A)$ — probability of observing evidence $B$ assuming $A$ is true.
  \item \textbf{Partition:} a set of mutually exclusive, exhaustive events $\{A_1,A_2,\dots,A_k\}$ that covers the sample space.
\end{itemize}

\section{Fundamental identities}
\sub\section{Multiplication rule}
Two equivalent forms:
\begin{align*}
  P(A\land B) &= P(A)\,P(B\mid A),\\
  P(A\land B) &= P(B)\,P(A\mid B).
\end{align*}

\sub\section{Conditional probability relation}
Provided $P(A)>0$,
\[
  P(B\mid A)=\frac{P(A\land B)}{P(A)}.
\]

\section{Bayes' Theorem (basic two-event form)}
\sub\section{Purpose}
Compute the posterior $P(A\mid B)$ from a prior $P(A)$ and the likelihood $P(B\mid A)$.

\sub\section{Formula}
For $P(B)\neq 0$,
\[
  P(A\mid B) = \frac{P(A)\,P(B\mid A)}{P(B)}.
\]
The denominator $P(B)$ is commonly computed via the law of total probability (next section).

\section{Law of total probability}
For a partition $\{A_1,\dots,A_k\}$ that is mutually exclusive and exhaustive:
\[
  P(B)=\sum_{i=1}^{k} P(A_i)\,P(B\mid A_i).
\]

\section{Generalized Bayes' Theorem (partition of $k$ events)}
For a partition $\{A_1,\dots,A_k\}$, and for any $j\in\{1,\dots,k\}$,
\[
  P(A_j\mid B)=\frac{P(A_j)\,P(B\mid A_j)}{\sum_{i=1}^{k} P(A_i)\,P(B\mid A_i)},
\]
provided the denominator $>0$. Requirements: the $A_i$ are mutually exclusive and exhaustive.

\section{Intuitive (frequency/table) method — practical solving steps}
\begin{enumerate}
  \item Assume a convenient total population $N$ (e.g., $1{,}000$ or $100{,}000$).
  \item Multiply priors $P(A_i)$ by $N$ to get counts for each $A_i$.
  \item Multiply each $A_i$ count by $P(B\mid A_i)$ to get counts for $A_i\cap B$.
  \item Sum the $A_i\cap B$ counts to obtain total \# of $B$.
  \item Compute posterior as $(A_j\cap B)/(\text{total }B)$.
\end{enumerate}
Advantage: reduces algebra mistakes; excellent for exams and quick checks.

\section{Short worked examples (to memorize pattern)}
\subsection*{Example A — ELT manufacturers}
Priors and likelihoods:
\[
\begin{array}{lcl}
P(A)=0.80,&\quad& P(D\mid A)=0.04,\\
P(B)=0.15,&\quad& P(D\mid B)=0.06,\\
P(C)=0.05,&\quad& P(D\mid C)=0.09.
\end{array}
\]
Posterior for $A$ given defective $D$:
\begin{align*}
  P(A\mid D)
  &= \frac{0.80\cdot 0.04}{0.80\cdot 0.04 + 0.15\cdot 0.06 + 0.05\cdot 0.09}\\
  &= \frac{0.032}{0.032 + 0.009 + 0.0045}
  = \frac{0.032}{0.0455}\approx 0.703.
\end{align*}

\subsection*{Example B — Cigar-smoking and sex (intuitive table)}
Priors and likelihoods:
\[
P(\text{Male})=0.51,\quad P(\text{Female})=0.49,
\]
\[
P(\text{Cigar}\mid \text{Male})=0.095,\quad P(\text{Cigar}\mid \text{Female})=0.017.
\]
Posterior $P(\text{Male}\mid \text{Cigar})$:
\begin{align*}
  \text{Numerator} &= 0.51\times 0.095 = 0.04845,\\
  \text{Denominator} &= 0.04845 + 0.49\times 0.017 = 0.04845 + 0.00833 = 0.05678,\\
  P(\text{Male}\mid \text{Cigar}) &= \frac{0.04845}{0.05678}\approx 0.854 \; (85.4\%).
\end{align*}

Table method (N=100,000):
\[
\begin{array}{lrr}
 & \text{Population} & \text{Cigar-smokers}\\
\midrule
\text{Males} & 51{,}000 & 51{,}000\times 0.095 = 4{,}845\\
\text{Females} & 49{,}000 & 49{,}000\times 0.017 = 833\\
\text{Total cigar} & 100{,}000 & 4{,}845 + 833 = 5{,}678
\end{array}
\]
Probability = $4{,}845 / 5{,}678 \approx 0.854$.

\section{Common applications and notes}
\begin{itemize}\setlength{\itemsep}{0.5em}
  \item \textbf{Diagnostic testing:} sensitivity $=P(\text{Test}+\mid \text{Disease})$, specificity $=P(\text{Test}-\mid \text{No disease})$. Positive predictive value $=P(\text{Disease}\mid \text{Test}+)$ is computed using Bayes' Theorem.
  \item Always check assumptions: the $\{A_i\}$ must be mutually exclusive and exhaustive when using the total-probability denominator.
  \item Denominator must be $>0$. If $P(B)=0$, then $P(A\mid B)$ is undefined.
  \item When priors are extreme (very small), even high likelihoods may produce small posteriors — particularly important in rare-disease contexts.
\end{itemize}

\section{Quick cheat-sheet (formulas to memorize)}
\begin{tcolorbox}[colback=blue!3!white,colframe=secondary,boxrule=0.7pt,sharp corners]\vspace{0.5em}
\begin{gather*}
P(A\land B)=P(A)P(B\mid A)=P(B)P(A\mid B),\\[4pt]
P(B)=\sum_{i} P(A_i)\,P(B\mid A_i)\quad\text{(law of total probability)},\\[4pt]
P(A_j\mid B)=\dfrac{P(A_j)\,P(B\mid A_j)}{\sum_{i} P(A_i)\,P(B\mid A_i)}\quad\text{(general Bayes)}.
\end{gather*}
\end{tcolorbox}

\section{Problem-solving checklist}
\begin{enumerate}
  \item Identify event of interest (posterior) and observed evidence $B$.
  \item Identify partition $\{A_i\}$ and priors $P(A_i)$.
  \item Determine likelihoods $P(B\mid A_i)$.
  \item Compute denominator via total probability or build a frequency table.
  \item Compute posterior $P(A_j\mid B)$.
  \item Check units, logic, and whether priors are mutually exclusive/exhaustive.
\end{enumerate}

% Final note box
\begin{tcolorbox}[colback=lightgray,colframe=primary,boxrule=0.8pt,sharp corners,breakable]\vspace{0.5em}
\sffamily\small
End of summary — use this sheet for quick revision and as a step-by-step guide when solving Bayes problems.
\end{tcolorbox}

\clearpage

% Optional appendix with quick reference (reiterated formulas)
\section*{Appendix: Notation reference}
\addcontentsline{toc}{section}{Appendix: Notation reference}
\begin{description}
  \item[$P(A)$] Prior probability of event $A$.
  \item[$P(B\mid A)$] Likelihood of evidence $B$ given $A$.
  \item[$P(A\mid B)$] Posterior probability of $A$ after observing $B$.
  \item[$P(A\land B)$ or $P(A\cap B)$] Joint probability of $A$ and $B$.
\end{description}

\vspace{2em}
\noindent\sffamily\small\color{accent}Document prepared for academic/course use. Source file: BayesTheorem.pdf.

\end{document}