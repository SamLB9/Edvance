
\documentclass[11pt,a4paper]{article}
\usepackage[utf8]{inputenc}
\usepackage[T1]{fontenc}
\usepackage{lmodern}
\usepackage{microtype}
\usepackage[a4paper,margin=25mm]{geometry}
\usepackage{xcolor}
\definecolor{primary}{HTML}{0B6FA4}
\definecolor{secondary}{HTML}{3A6B8A}
\definecolor{accent}{HTML}{6E7B8C}
\definecolor{lightgray}{HTML}{F3F5F7}
\usepackage{graphicx}
\usepackage{float}
\usepackage{amsmath,amssymb,amsthm}
\usepackage{mathtools}
\usepackage{enumitem}
\usepackage{booktabs}
\usepackage{array}
\usepackage{tcolorbox}
\tcbuselibrary{skins,breakable}
\usepackage{sectsty}
\allsectionsfont{\color{primary}\sffamily}
\usepackage{titlesec}
\titleformat{\section}{\Large\bfseries\sffamily\color{primary}}{\thesection}{1em}{}
\titleformat{\subsection}{\large\bfseries\sffamily\color{secondary}}{\thesubsection}{0.8em}{}
\titleformat{\subsubsection}{\normalsize\bfseries\sffamily\color{accent}}{\thesubsubsection}{0.6em}{}
\usepackage{fancyhdr}
\pagestyle{fancy}
\fancyhf{}
\renewcommand{\headrulewidth}{0.6pt}
\renewcommand{\footrulewidth}{0.4pt}
\renewcommand{\headrule}{\color{lightgray}\hrule height \headrulewidth \vspace{-\headrulewidth}}
\fancyhead[L]{\sffamily\small Course Notes Summary}
\fancyhead[C]{\sffamily\small Focus: Summary of main formulas please}
\fancyhead[R]{\sffamily\small Source: BayesTheorem.pdf}
\fancyfoot[L]{\sffamily\small Generated: 2025-09-05 16:06:08}
\fancyfoot[C]{}
\fancyfoot[R]{\sffamily\small \thepage}
\usepackage{hyperref}
\hypersetup{colorlinks=true, linkcolor=primary, urlcolor=secondary, citecolor=primary}
\setcounter{secnumdepth}{3}
\setcounter{tocdepth}{2}
\raggedbottom
\usepackage{titling}
\pretitle{\begin{center}\vspace*{1cm}\sffamily}
\posttitle{\par\end{center}\vskip 0.2em}
\preauthor{}\postauthor{}
\predate{}\postdate{}
% Helper macros used by model output
\newcommand{\Prob}{\mathrm{P}}
\newcommand{\given}{\,\mid\,}

\begin{document}


\begin{titlepage}
  \thispagestyle{empty}
  \begin{center}
    \vspace*{1cm}
    \includegraphics[width=0.3\textwidth]{16.png}
    \vspace{1em}
    {\sffamily\Huge\bfseries\color{primary} Course Notes Summary \par}
    \vspace{0.8em}
    {\sffamily\Large\color{secondary} Focus Area: Summary of main formulas please \par}
    \vspace{1.5em}
    \begin{tcolorbox}[enhanced,width=0.75\textwidth,colback=lightgray,colframe=primary,boxrule=0.8pt,sharp corners,left=6pt,right=6pt,top=6pt,bottom=6pt]\vspace{0.5em}
      \sffamily
      \begin{tabular}{@{}p{0.28\textwidth} p{0.62\textwidth}@{}}
        \textbf{Source: } & BayesTheorem.pdf \\
        \textbf{Generated:} & 2025-09-05 16:06:08 \\
        \textbf{Summary Type:} & Comprehensive \\
      \end{tabular}
    \end{tcolorbox}

    \vspace{1.5em}
    {\small\sffamily Prepared for quick revision and reference\par}
    \vfill

    {\small\sffamily \color{accent} Use this sheet as a step-by-step guide when solving problems.}
    \vspace{1.8cm}
  \end{center}
\end{titlepage}

\section{Key definitions and concepts}

\subsection{Definitions}
\begin{itemize}
\item Conditional probability: $P(B\mid A)$ = probability of $B$ given $A$ has occurred.
\item Prior probability: the initial probability of an event (before new information).
\item Posterior probability: the revised probability after new information.
\item Likelihood: $P(B\mid A)$, the probability of the observed evidence $B$ under hypothesis $A$.
\item Marginal (or total) probability: $P(B)$ = overall probability of $B$.
\item Joint probability: $P(A\text{ and }B)$ = probability both $A$ and $B$ occur.
\item Independence: $A$ and $B$ independent if $P(A\mid B)=P(A)$ (equivalently $P(A\text{ and }B)=P(A)P(B)$).
\end{itemize}

\subsection{Basic relationships}
\begin{itemize}
\item $P(B\mid A)=\dfrac{P(A\text{ and }B)}{P(A)}$
\item $P(A\text{ and }B)=P(A)P(B\mid A)=P(B)P(A\mid B)$
\item If $A$ and $B$ independent: $P(A\text{ and }B)=P(A)P(B)$
\end{itemize}

\section{Main formulas}

\subsection{Bayes' theorem (two events)}
$$
P(A\mid B)=\frac{P(B\mid A)P(A)}{P(B)}
$$

\subsection{Law of total probability (denominator for Bayes)}
For a partition $\{A_1,A_2,\dots,A_n\}$:
$$
P(B)=\sum_{i} P(B\mid A_i)P(A_i)
$$
Two-event special case:
$$
P(B)=P(B\mid A)P(A)+P(B\mid A^{c})P(A^{c})
$$

\subsection{Bayes' theorem (generalized for multiple events)}
For partition $\{A_1,\dots,A_n\}$:
$$
P(A_k\mid B)=\frac{P(B\mid A_k)P(A_k)}{\sum_{j} P(B\mid A_j)P(A_j)}
$$

\subsection{Odds form (useful for likelihood ratios)}
Posterior odds $=$ Likelihood ratio $\times$ Prior odds:
$$
\frac{P(A\mid B)}{P(A^{c}\mid B)}=\frac{P(B\mid A)}{P(B\mid A^{c})}\times\frac{P(A)}{P(A^{c})}
$$

\section{Computation methods and examples}

\subsection{Intuitive (frequency/table) method --- steps}
\begin{enumerate}
\item Assume a convenient total $N$ (e.g., $N=100{,}000$).
\item Convert probabilities to counts by multiplying $N$.
\item Fill a contingency table with counts for each combination.
\item Compute posterior as $\dfrac{\text{count}(A\text{ and }B)}{\text{count}(B)}$.
\end{enumerate}

\subsection{Example: cigar smoking (two-category Bayes via table)}
\begin{itemize}
\item Given: $P(\text{male})=0.51$, $P(\text{female})=0.49$, $P(\text{cigar}\mid\text{male})=0.095$, $P(\text{cigar}\mid\text{female})=0.017$.
\item Assume $N=100{,}000$.
\item Males $=51{,}000$. Cigar-smoking males $=0.095\times 51{,}000=4{,}845$.
\item Females $=49{,}000$. Cigar-smoking females $=0.017\times 49{,}000=833$.
\item Total cigar smokers $=4{,}845+833=5{,}678$.
\item $P(\text{male}\mid\text{cigar})=\dfrac{4{,}845}{5{,}678}\approx 0.853$.
\end{itemize}

\subsection{Example: defective ELT (generalized Bayes)}
\begin{itemize}
\item Given: $P(A)=0.80$, $P(B)=0.15$, $P(C)=0.05$.
\item Defect rates: $P(D\mid A)=0.04$, $P(D\mid B)=0.06$, $P(D\mid C)=0.09$.
\item Numerator for $P(A\mid D)$:
$$
P(D\mid A)P(A)=0.04\times 0.80=0.032
$$
\item Denominator:
$$
0.032+0.06\times 0.15+0.09\times 0.05=0.032+0.009+0.0045=0.0455
$$
\item $P(A\mid D)=\dfrac{0.032}{0.0455}\approx 0.703$.
\end{itemize}

\section{Important relationships and connections}
\begin{itemize}
\item Bayes requires the law of total probability to compute $P(B)$ (denominator).
\item Posterior becomes new prior for sequential updating with additional evidence.
\item Independence simplifies conditional expressions and Bayes calculations.
\item Likelihood ratios $\bigl(P(B\mid A)/P(B\mid A^{c})\bigr)$ quantify how evidence shifts beliefs.
\item Frequency/table method avoids algebraic substitution errors and aids intuition.
\end{itemize}

\section{Quick reference (compact formulas)}
\begin{itemize}
\item Conditional: $P(B\mid A)=\dfrac{P(A\text{ and }B)}{P(A)}$
\item Joint: $P(A\text{ and }B)=P(A)P(B\mid A)=P(B)P(A\mid B)$
\item Law of total probability: $P(B)=\sum_{i} P(B\mid A_i)P(A_i)$
\item Bayes (two events): $P(A\mid B)=\dfrac{P(B\mid A)P(A)}{P(B)}$
\item Bayes (multiple): $P(A_k\mid B)=\dfrac{P(B\mid A_k)P(A_k)}{\sum_{j} P(B\mid A_j)P(A_j)}$
\item Odds form: $\dfrac{P(A\mid B)}{P(A^{c}\mid B)}=\dfrac{P(B\mid A)}{P(B\mid A^{c})}\times\dfrac{P(A)}{P(A^{c})}$
\end{itemize}

\end{document}
