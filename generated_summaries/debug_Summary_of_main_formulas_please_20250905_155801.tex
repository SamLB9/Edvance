
\documentclass[11pt,a4paper]{article}
\usepackage[utf8]{inputenc}
\usepackage[T1]{fontenc}
\usepackage{lmodern}
\usepackage{microtype}
\usepackage[a4paper,margin=25mm]{geometry}
\usepackage{xcolor}
\definecolor{primary}{HTML}{0B6FA4}
\definecolor{secondary}{HTML}{3A6B8A}
\definecolor{accent}{HTML}{6E7B8C}
\definecolor{lightgray}{HTML}{F3F5F7}
\usepackage{graphicx}
\usepackage{float}
\usepackage{amsmath,amssymb,amsthm}
\usepackage{mathtools}
\usepackage{enumitem}
\usepackage{booktabs}
\usepackage{array}
\usepackage{tcolorbox}
\tcbuselibrary{skins,breakable}
\usepackage{sectsty}
\allsectionsfont{\color{primary}\sffamily}
\usepackage{titlesec}
\titleformat{\section}{\Large\bfseries\sffamily\color{primary}}{\thesection}{1em}{}
\titleformat{\subsection}{\large\bfseries\sffamily\color{secondary}}{\thesubsection}{0.8em}{}
\titleformat{\subsubsection}{\normalsize\bfseries\sffamily\color{accent}}{\thesubsubsection}{0.6em}{}
\usepackage{fancyhdr}
\pagestyle{fancy}
\fancyhf{}
\renewcommand{\headrulewidth}{0.6pt}
\renewcommand{\footrulewidth}{0.4pt}
\renewcommand{\headrule}{\color{lightgray}\hrule height \headrulewidth \vspace{-\headrulewidth}}
\fancyhead[L]{\sffamily\small Course Notes Summary}
\fancyhead[C]{\sffamily\small Focus: Summary of main formulas please}
\fancyhead[R]{\sffamily\small Source: BayesTheorem.pdf}
\fancyfoot[L]{\sffamily\small Generated: 2025-09-05 15:58:01}
\fancyfoot[C]{}
\fancyfoot[R]{\sffamily\small \thepage}
\usepackage{hyperref}
\hypersetup{colorlinks=true, linkcolor=primary, urlcolor=secondary, citecolor=primary}
\setcounter{secnumdepth}{3}
\setcounter{tocdepth}{2}
\raggedbottom
\usepackage{titling}
\pretitle{\begin{center}\vspace*{1cm}\sffamily}
\posttitle{\par\end{center}\vskip 0.2em}
\preauthor{}\postauthor{}
\predate{}\postdate{}
% Helper macros used by model output
\newcommand{\Prob}{\mathrm{P}}
\newcommand{\given}{\,\mid\,}

\begin{document}


\begin{titlepage}
  \thispagestyle{empty}
  \begin{center}
    \vspace*{1cm}
    \includegraphics[width=0.3\textwidth]{16.png}
    \vspace{1em}
    {\sffamily\Huge\bfseries\color{primary} Course Notes Summary \par}
    \vspace{0.8em}
    {\sffamily\Large\color{secondary} Focus Area: Summary of main formulas please \par}
    \vspace{1.5em}
    \begin{tcolorbox}[enhanced,width=0.75\textwidth,colback=lightgray,colframe=primary,boxrule=0.8pt,sharp corners,left=6pt,right=6pt,top=6pt,bottom=6pt]\vspace{0.5em}
      \sffamily
      \begin{tabular}{@{}p{0.28\textwidth} p{0.62\textwidth}@{}}
        \textbf{Source: } & BayesTheorem.pdf \\
        \textbf{Generated:} & 2025-09-05 15:58:01 \\
        \textbf{Summary Type:} & Comprehensive \\
      \end{tabular}
    \end{tcolorbox}

    \vspace{1.5em}
    {\small\sffamily Prepared for quick revision and reference\par}
    \vfill

    {\small\sffamily \color{accent} Use this sheet as a step-by-step guide when solving problems.}
    \vspace{1.8cm}
  \end{center}
\end{titlepage}

\section{Key definitions and concepts}
\begin{itemize}
\item Conditional probability: $P(B\mid A)$ is the probability $B$ occurs given $A$ has occurred.
\item Prior probability: initial probability before new information.
\item Posterior probability: revised probability after incorporating new information.
\item Likelihood: $P(\text{new data}\mid\text{hypothesis})$, used to update the prior.
\item Joint probability: $P(A\cap B)$ is the probability both $A$ and $B$ occur.
\item Law of total probability: expresses $P(A)$ via a partition of the sample space.
\end{itemize}

\section{Fundamental formulas}
\begin{itemize}
\item Conditional probability (definition)
\item $$P(B\mid A)=\frac{P(A\cap B)}{P(A)}$$
\item Symmetric expression for joint probability
\item $$P(A\cap B)=P(A)\,P(B\mid A)=P(B)\,P(A\mid B)$$
\item Bayes' theorem (basic two-event form)
\item $$P(B\mid A)=\frac{P(A\mid B)\,P(B)}{P(A)}$$
\item Law of total probability (for a partition $B_1,\dots,B_n$)
\item $$P(A)=\sum_{i=1}^n P(A\mid B_i)\,P(B_i)$$
\item Bayes' theorem (generalized for a partition $B_1,\dots,B_n$)
\item $$P(B_j\mid A)=\frac{P(A\mid B_j)\,P(B_j)}{\displaystyle\sum_{i=1}^n P(A\mid B_i)\,P(B_i)}$$
\end{itemize}

\section{Important relationships and notes}
\begin{itemize}
\item Prior $\to$ Likelihood $\to$ Posterior sequence:
\begin{enumerate}
\item Start with prior $P(B_j)$.
\item Use likelihood $P(A\mid B_j)$.
\item Compute posterior $P(B_j\mid A)$ via Bayes' theorem.
\end{enumerate}
\item Normalizing constant in Bayes' rule is $P(A)$ given by the law of total probability.
\item $P(B\mid A)$ increases when $P(A\mid B)$ is relatively large compared with $P(A\mid\text{other events})$ weighted by priors.
\item Bayes can be applied with discrete counts or probabilities; both formulations are equivalent.
\end{itemize}

\section{Intuitive (frequency/table) method}
\begin{itemize}
\item Method:
\begin{enumerate}
\item Assume a convenient total $N$ (e.g., $1000$ or $100000$).
\item Convert probabilities to counts: $\text{count}(B_i)=N\cdot P(B_i)$.
\item Compute cells: $\text{count}(A\cap B_i)=\text{count}(B_i)\cdot P(A\mid B_i)$.
\item Use counts to compute desired conditional:
$$P(B_j\mid A)=\frac{\text{count}(A\cap B_j)}{\sum_i \text{count}(A\cap B_i)}.$$
\end{enumerate}
\item Advantage: reduces algebra errors, makes denominators explicit.
\end{itemize}

\section{Worked examples}
\begin{itemize}
\item Example A: Orange County cigar smokers
\begin{itemize}
\item Given:
\begin{itemize}
\item $P(\text{male})=0.51$
\item $P(\text{female})=0.49$
\item $P(\text{cigar}\mid\text{male})=0.095$
\item $P(\text{cigar}\mid\text{female})=0.017$
\end{itemize}
\item Bayes calculation:
\[
P(\text{male}\mid\text{cigar})
=\frac{P(\text{cigar}\mid\text{male})\,P(\text{male})}
{P(\text{cigar}\mid\text{male})\,P(\text{male})+P(\text{cigar}\mid\text{female})\,P(\text{female})}
=\frac{0.095\cdot 0.51}{0.095\cdot 0.51 + 0.017\cdot 0.49}
\]
\[
=\frac{0.04845}{0.04845+0.00833}\approx\frac{0.04845}{0.05678}\approx 0.853
\]
\item Frequency/table check ( $N=100{,}000$):
\begin{itemize}
\item males $=51{,}000$; male cigar $=0.095\times 51{,}000=4{,}845$
\item females $=49{,}000$; female cigar $=0.017\times 49{,}000=833$
\item total cigar $=4{,}845+833=5{,}678$
\item $P(\text{male}\mid\text{cigar})=\dfrac{4{,}845}{5{,}678}\approx 0.853$
\end{itemize}
\end{itemize}

\item Example B: ELT manufacturers and defectives
\begin{itemize}
\item Given:
\begin{itemize}
\item $P(A)=0.80,\quad P(B)=0.15,\quad P(C)=0.05$
\item $P(\text{defective}\mid A)=0.04,\quad P(\text{defective}\mid B)=0.06,\quad P(\text{defective}\mid C)=0.09$
\end{itemize}
\item Bayes calculation:
\[
P(A\mid \text{defective})
=\frac{P(\text{defective}\mid A)\,P(A)}
{P(\text{defective}\mid A)P(A)+P(\text{defective}\mid B)P(B)+P(\text{defective}\mid C)P(C)}
=\frac{0.04\cdot 0.80}{0.04\cdot 0.80 + 0.06\cdot 0.15 + 0.09\cdot 0.05}
\]
\[
=\frac{0.032}{0.032 + 0.009 + 0.0045}=\frac{0.032}{0.0455}\approx 0.703
\]
\item Frequency/table check ($N=10{,}000$):
\begin{itemize}
\item $A$ total $=8{,}000$; defective from $A=0.04\times 8{,}000=320$
\item $B$ total $=1{,}500$; defective from $B=0.06\times 1{,}500=90$
\item $C$ total $=500$; defective from $C=0.09\times 500=45$
\item total defective $=320+90+45=455$
\item $P(A\mid\text{defective})=\dfrac{320}{455}\approx 0.703$
\end{itemize}
\end{itemize}
\end{itemize}

\section{Quick solution checklist (ordered steps)}
\begin{enumerate}
\item Identify hypothesis events $B_1,\dots,B_n$ and the observed event $A$.
\item Record priors $P(B_i)$ and likelihoods $P(A\mid B_i)$.
\item Compute $P(A)=\sum_i P(A\mid B_i)\,P(B_i)$.
\item Compute posterior for target $B_j$:
\[
P(B_j\mid A)=\frac{P(A\mid B_j)\,P(B_j)}{P(A)}.
\]
\item Optionally verify with frequency/table method using a convenient $N$.
\end{enumerate}

\section{Compact reference formulas (for quick revision)}
\begin{itemize}
\item $$P(B\mid A)=\frac{P(A\mid B)\,P(B)}{P(A)}$$
\item $$P(A)=\sum_{i} P(A\mid B_i)\,P(B_i)$$
\item $$P(A\cap B)=P(A)\,P(B\mid A)=P(B)\,P(A\mid B)$$
\end{itemize}

\end{document}
