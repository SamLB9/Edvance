
\documentclass[11pt,a4paper]{article}
\usepackage[utf8]{inputenc}
\usepackage[T1]{fontenc}
\usepackage{lmodern}
\usepackage{microtype}
\usepackage[a4paper,margin=25mm]{geometry}
\usepackage{xcolor}
\definecolor{primary}{HTML}{0B6FA4}
\definecolor{secondary}{HTML}{3A6B8A}
\definecolor{accent}{HTML}{6E7B8C}
\definecolor{lightgray}{HTML}{F3F5F7}
\usepackage{graphicx}
\usepackage{float}
\usepackage{amsmath,amssymb,amsthm}
\usepackage{mathtools}
\usepackage{enumitem}
\usepackage{booktabs}
\usepackage{array}
\usepackage{tcolorbox}
\tcbuselibrary{skins,breakable}
\usepackage{sectsty}
\allsectionsfont{\color{primary}\sffamily}
\usepackage{titlesec}
\titleformat{\section}{\Large\bfseries\sffamily\color{primary}}{\thesection}{1em}{}
\titleformat{\subsection}{\large\bfseries\sffamily\color{secondary}}{\thesubsection}{0.8em}{}
\titleformat{\subsubsection}{\normalsize\bfseries\sffamily\color{accent}}{\thesubsubsection}{0.6em}{}
\usepackage{fancyhdr}
\pagestyle{fancy}
\fancyhf{}
\renewcommand{\headrulewidth}{0.6pt}
\renewcommand{\footrulewidth}{0.4pt}
\renewcommand{\headrule}{\color{lightgray}\hrule height \headrulewidth \vspace{-\headrulewidth}}
\fancyhead[L]{\sffamily\small Course Notes Summary}
\fancyhead[C]{\sffamily\small Focus: Summary of main formulas please}
\fancyhead[R]{\sffamily\small Source: BayesTheorem.pdf}
\fancyfoot[L]{\sffamily\small Generated: 2025-09-05 15:21:07}
\fancyfoot[C]{}
\fancyfoot[R]{\sffamily\small \thepage}
\usepackage{hyperref}
\hypersetup{colorlinks=true, linkcolor=primary, urlcolor=secondary, citecolor=primary}
\setcounter{secnumdepth}{3}
\setcounter{tocdepth}{2}
\raggedbottom
\usepackage{titling}
\pretitle{\begin{center}\vspace*{1cm}\sffamily}
\posttitle{\par\end{center}\vskip 0.2em}
\preauthor{}\postauthor{}
\predate{}\postdate{}
% Helper macros used by model output
\newcommand{\Prob}{\mathrm{P}}
\newcommand{\given}{\,\mid\,}

\begin{document}

\begin{titlepage}
  \thispagestyle{empty}
  \begin{center}
    \vspace*{1cm}
    {\sffamily\Huge\bfseries\color{primary} Course Notes Summary \par}
    \vspace{0.8em}
    {\sffamily\Large\color{secondary} Focus Area: Summary of main formulas please \par}
    \vspace{1.5em}
    \begin{tcolorbox}[enhanced,width=0.75\textwidth,colback=lightgray,colframe=primary,boxrule=0.8pt,sharp corners,left=6pt,right=6pt,top=6pt,bottom=6pt]\vspace{0.5em}
      \sffamily
      \begin{tabular}{@{}p{0.28\textwidth} p{0.62\textwidth}@{}}
        \textbf{Source: } & BayesTheorem.pdf \\
        \textbf{Generated:} & 2025-09-05 15:21:07 \\
        \textbf{Summary Type:} & Comprehensive \\
      \end{tabular}
    \end{tcolorbox}

    \vspace{1.5em}
    {\small\sffamily Prepared for quick revision and reference\par}
    \vfill

    {\small\sffamily \color{accent} Use this sheet as a step-by-step guide when solving problems.}
    \vspace{1.8cm}
  \end{center}
\end{titlepage}

Summary of Main Formulas — Bayes’ Theorem (study / revision sheet)

1. Key definitions
- Conditional probability: P(B | A) = probability that B occurs given A has occurred.
- Prior probability: initial probability of an event before new information is used (denote P(A)).
- Likelihood: P(new information | hypothesis) — e.g., P(B | A).
- Posterior probability: revised probability after using new information — e.g., P(A | B).
- Evidence (or marginal likelihood): overall probability of the new information, P(B).

2. Fundamental formulas
- Conditional probability (basic definition)
  P(B | A) = P(A ∩ B) / P(A), provided P(A) > 0.

- Multiplication rule (from conditional probability)
  P(A ∩ B) = P(A) P(B | A) = P(B) P(A | B).

- Independence (special case)
  A and B independent ⇔ P(B | A) = P(B) ⇔ P(A ∩ B) = P(A) P(B).

3. Bayes’ theorem — two-event form
- Bayes’ formula (general two-event)
  P(A | B) = [P(B | A) P(A)] / P(B).

- Using the complement Aᶜ (two-category partition)
  P(A | B) = [P(B | A) P(A)] / [P(B | A) P(A) + P(B | Aᶜ) P(Aᶜ)].

- Mnemonic form
  posterior ∝ prior × likelihood
  (then normalize by dividing by evidence P(B)).

4. Total probability theorem (needed for denominator / evidence)
- For a partition {A1, A2, …, An} of the sample space,
  P(B) = Σ_j P(B | A_j) P(A_j).
- For two events A and Aᶜ: P(B) = P(B | A)P(A) + P(B | Aᶜ)P(Aᶜ).

5. Bayes’ theorem — generalized (n events / partition)
- For events A1, A2, …, An forming a partition,
  P(A_i | B) = [P(A_i) P(B | A_i)] / Σ_{j=1..n} [P(A_j) P(B | A_j)].
- Each posterior P(A_i | B) is nonnegative and Σ_i P(A_i | B) = 1.

6. Intuitive (frequency/table) method — practical solving tip
- Pick a convenient total N (e.g., 1000, 10,000, 100,000).
- Convert each prior P(A_i) to counts: count(A_i) = N × P(A_i).
- Convert likelihoods to counts: count(A_i ∩ B) = count(A_i) × P(B | A_i).
- Sum counts of B across all Ai to get total count(B) = N × P(B).
- Posterior P(A_i | B) = count(A_i ∩ B) / count(B).
- This avoids algebra errors and gives intuitive contingency tables.

7. Step-by-step procedure for a Bayes problem
1. Identify hypotheses/partitions A1, A2, … (priors P(A_i)).
2. Identify observed evidence B and likelihoods P(B | A_i).
3. Compute evidence P(B) using total probability: Σ P(A_i) P(B | A_i).
4. Compute posterior(s): P(A_i | B) = [P(A_i) P(B | A_i)] / P(B).
5. Optionally, construct counts via an assumed N to check numerically.

8. Short worked examples (compact)

- Example A: Cigar-smoking / gender (from text)
  Given: P(Male)=0.51, P(Female)=0.49, P(Cigar | Male)=0.095, P(Cigar | Female)=0.017.
  Compute P(Male | Cigar):
  P(Cigar) = 0.51×0.095 + 0.49×0.017 = 0.04845 + 0.00833 = 0.05678.
  P(Male | Cigar) = (0.51×0.095) / 0.05678 ≈ 0.04845 / 0.05678 ≈ 0.853 (≈85.3%).

- Example B: ELT manufacturers and defective units (three-category)
  Given: P(A)=0.80, P(B)=0.15, P(C)=0.05; defect rates P(D | A)=0.04, P(D | B)=0.06, P(D | C)=0.09.
  P(D) = 0.80×0.04 + 0.15×0.06 + 0.05×0.09 = 0.032 + 0.009 + 0.0045 = 0.0455.
  P(A | D) = (0.80×0.04) / 0.0455 = 0.032 / 0.0455 ≈ 0.703 (≈70.3%).
  (Alternative counts: assume N=10,000 → defective counts: A:320, B:90, C:45 → posterior = 320/455 ≈ 0.703.)

9. Important relationships & checks
- Consistency: P(A ∩ B) computed either way must match: P(A)P(B | A) = P(B)P(A | B).
- Posterior normalization: Σ_i P(A_i | B) = 1 when {A_i} is a partition.
- Edge case (zero denominators): Bayes’ formula requires P(B) > 0.
- Independence test: if P(A | B) = P(A), then B gives no information about A.
- Beware base-rate fallacy: high likelihood P(B | A) alone does not imply high posterior P(A | B) — priors matter.

10. Common pitfalls and tips
- Always identify priors and likelihoods explicitly before algebra.
- Use the total-probability denominator — forgetting terms is the common error.
- Use frequency tables for intuition and arithmetic checks.
- Round only at the final step to avoid cumulative rounding errors.
- For small probabilities or many events, use precise arithmetic or software.

11. Quick formula cheat-sheet (memorize these)
- P(B | A) = P(A ∩ B) / P(A)
- P(A ∩ B) = P(A) P(B | A)
- P(B) = Σ_j P(B | A_j) P(A_j)
- P(A_i | B) = [P(A_i) P(B | A_i)] / Σ_j [P(A_j) P(B | A_j)]

Use this sheet to:
- Set up Bayes problems quickly (identify priors, likelihoods).
- Convert to counts for sanity checks.
- Remember posterior ∝ prior × likelihood and then normalize.

If you want, I can convert this into a one-page printable study card or produce a few more worked examples with contingency tables.

\end{document}
