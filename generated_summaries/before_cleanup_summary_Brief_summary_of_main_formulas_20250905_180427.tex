\section{Definitions}
\begin{itemize}
\item Prior probability: initial probability of an event before new information is observed.
\item Posterior probability: revised probability of an event after new information is observed.
\item Conditional probability: probability of event $B$ given event $A$ has occurred, written $P(B\mid A)$.
\end{itemize}

\section{Main formulas}
\begin{itemize}
\item Conditional probability (basic):
$$
P(B\mid A)=\frac{P(A\text{ and }B)}{P(A)}.
$$
\item Bayes' theorem (two events):
$$
P(B\mid A)=\frac{P(A\mid B)\,P(B)}{P(A)}.
$$
\item Total probability (two-event form; partition $B,B^c$):
$$
P(A)=P(B)P(A\mid B)+P(B^c)P(A\mid B^c),
$$
(use appropriate partitioning; more generally below).
\item Bayes' theorem (generalized, partition $A_1,\dots,A_n$):
$$
P(A_k\mid B)=\frac{P(A_k)\,P(B\mid A_k)}{\displaystyle\sum_{j=1}^n P(A_j)\,P(B\mid A_j)}.
$$
\item Total probability (generalized):
$$
P(B)=\sum_{j=1}^n P(A_j)\,P(B\mid A_j).
$$
\end{itemize}

\section{Intuitive/frequency formulation}
\begin{itemize}
\item Replace probabilities by counts using an assumed total $N$.
\item Compute cell counts:
$$
\text{count}(A_j\text{ and }B)=N\cdot P(A_j)\cdot P(B\mid A_j).
$$
\item Then
$$
P(A_k\mid B)=\frac{\text{count}(A_k\text{ and }B)}{\sum_j \text{count}(A_j\text{ and }B)}.
$$
\item This produces the same numeric result as Bayes' formula and helps avoid substitution errors.
\end{itemize}

\section{Examples}
\subsection{ELT manufacturers (three events)}
Given:
\begin{itemize}
\item $P(A)=0.80$, $P(B)=0.15$, $P(C)=0.05$.
\item $P(D\mid A)=0.04$, $P(D\mid B)=0.06$, $P(D\mid C)=0.09$.
\end{itemize}
Compute $P(A\mid D)$:
$$
P(A\mid D)=\frac{P(A)\,P(D\mid A)}{P(A)P(D\mid A)+P(B)P(D\mid B)+P(C)P(D\mid C)}
$$
$$
=\frac{0.80\cdot 0.04}{0.80\cdot 0.04 + 0.15\cdot 0.06 + 0.05\cdot 0.09}
=\frac{0.032}{0.032 + 0.009 + 0.0045}
=\frac{0.032}{0.0455}\approx 0.703.
$$
Frequency check ($N=10{,}000$): defective counts = $320$ (A), $90$ (B), $45$ (C), total $455$. $P=320/455\approx 0.703$.

\subsection{Orange County cigar example}
Given:
\begin{itemize}
\item $P(\text{male})=0.51$, $P(\text{female})=0.49$.
\item $P(\text{cigar}\mid \text{male})=0.095$, $P(\text{cigar}\mid \text{female})=0.017$.
\end{itemize}
Compute $P(\text{male}\mid \text{cigar})$:
$$
P(\text{cigar})=0.51\cdot 0.095 + 0.49\cdot 0.017 = 0.04845 + 0.00833 = 0.05678,
$$
$$
P(\text{male}\mid \text{cigar})=\frac{0.04845}{0.05678}\approx 0.853.
$$
Frequency check ($N=100{,}000$): male smokers $=4845$, female smokers $=833$, total smokers $=5678$. $P=4845/5678\approx 0.853$.

\section{Application steps (how to use Bayes' theorem)}
\begin{enumerate}
\item Identify the hypothesis events $A_1,\dots,A_n$ that form a partition of the sample space.
\item Determine prior probabilities $P(A_j)$.
\item Determine likelihoods $P(B\mid A_j)$ for the observed evidence $B$.
\item Compute $P(B)$ via total probability: $P(B)=\sum_j P(A_j)\,P(B\mid A_j)$.
\item Compute posterior: $P(A_k\mid B)=\dfrac{P(A_k)\,P(B\mid A_k)}{P(B)}$.
\end{enumerate}

\section{Important relationships and connections}
\begin{itemize}
\item Posterior \propto Prior $\times$ Likelihood:
$$
P(A_k\mid B)\propto P(A_k)\,P(B\mid A_k).
$$
\item Total probability provides the normalizing constant (denominator) for Bayes' theorem.
\item The frequency-table (count) method is algebraically equivalent to the formula method but often easier to implement and check by inspection.
\item Bayes' theorem is sequential: posteriors from one update can serve as priors for the next update when new evidence arrives.
\end{itemize}

\section{Notes and pitfalls}
\begin{itemize}
\item Ensure the $A_j$ are a complete partition and $P(B)>0$.
\item Distinguish carefully between $P(B\mid A)$ and $P(A\mid B)$; they are generally not equal.
\item Use the frequency/table method to reduce algebraic substitution errors.
\item Round only at the final step to avoid cumulative rounding error.
\end{itemize}

\section{Quick reference (compact formulas)}
\begin{itemize}
\item $P(B\mid A)=\dfrac{P(A\text{ and }B)}{P(A)}$.
\item $P(B\mid A)=\dfrac{P(A\mid B)\,P(B)}{P(A)}$.
\item $P(B)=\sum_j P(A_j)\,P(B\mid A_j)$.
\item $P(A_k\mid B)=\dfrac{P(A_k)\,P(B\mid A_k)}{\displaystyle\sum_j P(A_j)\,P(B\mid A_j)}$.
\end{itemize}