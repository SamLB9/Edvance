\section{Overview}

\subsection{Purpose}
\begin{itemize}
\item Bayes' theorem revises prior probabilities when new evidence is observed.
\item It converts likelihoods and priors into posterior probabilities.
\end{itemize}

\subsection{Key definitions}
\begin{itemize}
\item Prior probability: $P(H)$ — probability assigned to hypothesis $H$ before new evidence.
\item Likelihood: $P(E\mid H)$ — probability of observing evidence $E$ if $H$ is true.
\item Marginal (evidence) probability: $P(E)$ — total probability of evidence under all hypotheses.
\item Posterior probability: $P(H\mid E)$ — revised probability of $H$ after observing $E$.
\end{itemize}

\section{Core formulas}
\begin{itemize}
\item Conditional probability definition:
$$
P(A\mid B)=\frac{P(A\land B)}{P(B)}.
$$
\item Bayes' theorem (two hypotheses):
$$
P(A\mid B)=\frac{P(B\mid A)\,P(A)}{P(B\mid A)\,P(A)+P(B\mid \lnot A)\,P(\lnot A)}.
$$
\item Bayes' theorem (general, partition $\{A_1,\dots,A_n\}$):
$$
P(A_i\mid E)=\frac{P(A_i)\,P(E\mid A_i)}{\sum_j P(A_j)\,P(E\mid A_j)}.
$$
\item Extended example form (four hypotheses):
$$
P(A\mid Z)=\frac{P(A)P(Z\mid A)}{P(A)P(Z\mid A)+P(B)P(Z\mid B)+P(C)P(Z\mid C)+P(D)P(Z\mid D)}.
$$
\end{itemize}

\section{Procedure to apply Bayes' theorem}
\begin{enumerate}
\item Identify hypothesis $H$ and evidence $E$.
\item Obtain priors $P(H)$ and $P(\lnot H)$ (or $P(A_i)$ for all hypotheses).
\item Obtain likelihoods $P(E\mid H)$ and $P(E\mid \lnot H)$ (or $P(E\mid A_i)$ for all).
\item Compute marginal $P(E)=\sum_j P(A_j)P(E\mid A_j)$.
\item Compute posterior:
$$
P(H\mid E)=\frac{P(E\mid H)\,P(H)}{P(E)}.
$$
\item Interpret result: posterior replaces prior given the evidence.
\end{enumerate}

\section{Worked examples (from PDF)}

\subsection{Orange County cigar example (credit-card survey)}
\begin{itemize}
\item Given: $P(\text{male})=0.51$, $P(\text{female})=0.49$.
\item Likelihoods: $P(\text{smoke}\mid\text{male})=0.095$, $P(\text{smoke}\mid\text{female})=0.017$.
\item Posterior:
$$
P(\text{male}\mid\text{smoke})=\frac{0.51\cdot 0.095}{0.51\cdot 0.095+0.49\cdot 0.017}\approx 0.853.
$$
\item Interpretation: Observing cigar smoking increases probability subject is male.
\end{itemize}

\subsection{Pregnancy test (contingency table)}
\begin{itemize}
\item Table totals:
  \begin{itemize}
  \item Pregnant \& positive: $80$
  \item Pregnant \& negative: $5$
  \item Not pregnant \& positive: $3$
  \item Not pregnant \& negative: $11$
  \item Totals: Pregnant $=85$, Not pregnant $=14$, Positive $=83$, Negative $=16$, $N=99$
  \end{itemize}
\item Results:
  \begin{itemize}
  \item $P(\text{pregnant})=\dfrac{85}{99}=0.859$.
  \item $P(\text{pregnant}\mid\text{positive})=\dfrac{80}{83}=0.964$.
  \item $P(\text{not pregnant})=\dfrac{14}{99}=0.141$.
  \item $P(\text{not pregnant}\mid\text{negative})=\dfrac{11}{16}=0.6875$.
  \end{itemize}
\end{itemize}

\subsection{Pleas and sentences}
\begin{itemize}
\item Given: $P(\text{sent to prison})=0.45$, $P(\text{not sent})=0.55$.
\item Likelihoods: $P(\text{guilty}\mid\text{prison})=0.40$, $P(\text{guilty}\mid\text{not prison})=0.55$.
\item Posteriors:
  \begin{itemize}
  \item Prior: $P(\text{not sent})=0.55$.
  \item $P(\text{not sent}\mid\text{guilty})=\dfrac{0.55\cdot 0.55}{0.55\cdot 0.55+0.45\cdot 0.40}\approx 0.627$.
  \item Complement: $P(\text{sent}\mid\text{guilty})=1-0.627=0.373$.
  \end{itemize}
\end{itemize}

\subsection{HIV screening (at-risk population)}
\begin{itemize}
\item Given: $P(\text{HIV})=0.10$.
\item Test correct $95\%$ of the time (assumed):
  \begin{itemize}
  \item $P(+\mid\text{HIV})=0.95$ (sensitivity).
  \item $P(+\mid\text{not HIV})=0.05$ (false positive rate).
  \end{itemize}
\item Posterior if test positive:
$$
P(\text{HIV}\mid +)=\frac{0.10\cdot 0.95}{0.10\cdot 0.95+0.90\cdot 0.05}\approx 0.679.
$$
\item Posterior if test negative:
$$
P(\text{HIV}\mid -)=\frac{0.10\cdot 0.05}{0.10\cdot 0.05+0.90\cdot 0.95}\approx 0.0058.
$$
\end{itemize}

\subsection{Emergency Locator Transmitters (ELT) — method}
\begin{itemize}
\item Given data referenced in Exercise 4 (priors by manufacturer and defect rates).
\item Example answers provided:
  \begin{itemize}
  \item $P(\text{selected ELT from Chartair})=0.05$.
  \item $P(\text{Chartair}\mid\text{defective})=0.0989$.
  \end{itemize}
\item Method: treat manufacturers as partition $\{M_1,\dots,M_k\}$ and apply the general formula.
\end{itemize}

\section{Important relationships and connections}
\begin{itemize}
\item Posterior depends on both prior and likelihood; a small prior can be overwhelmed by strong likelihood or vice versa.
\item Marginal likelihood $P(E)$ normalizes the posterior; it aggregates contributions from all competing hypotheses.
\item Bayes' theorem connects conditional probabilities $P(A\mid B)$ and $P(B\mid A)$.
\item When tests are imperfect, high sensitivity and low prevalence can yield moderate post-test positive probabilities (illustrated by HIV example).
\item Contingency tables give direct counts to compute priors, likelihoods, marginals, and posteriors.
\end{itemize}

\section{Compact worked formulas and reminders}
\begin{itemize}
\item $P(A\mid B)=\dfrac{P(B\mid A)\,P(A)}{\sum_j P(B\mid A_j)\,P(A_j)}$.
\item Marginal: $P(B)=\sum_j P(B\mid A_j)\,P(A_j)$.
\item Complement checks: $P(A\mid B)+P(\lnot A\mid B)=1$.
\item When given counts: convert counts to probabilities by dividing by total $N$.
\end{itemize}

\section{Sample answers from exercises (selected)}
\begin{itemize}
\item Pregnancy: $P(\text{pregnant})=\dfrac{85}{99}=0.859$; $P(\text{pregnant}\mid\text{positive})=\dfrac{80}{83}=0.964$.
\item Orange County prior female: $0.49$; $P(\text{female}\mid\text{smokes cigar})\approx 0.147$ (complement example).
\item ELT example: $P(\text{Chartair})=0.05$; $P(\text{Chartair}\mid\text{defective})=0.0989$.
\item Pleas example: $P(\text{not sent})=0.55$; $P(\text{not sent}\mid\text{guilty})=0.627$.
\item HIV positive test posterior: $0.679$.
\end{itemize}

\section{Quick revision checklist}
\begin{enumerate}
\item Identify hypotheses and evidence.
\item Write priors $P(A_i)$.
\item Write likelihoods $P(E\mid A_i)$.
\item Compute marginal $P(E)=\sum_i P(E\mid A_i)P(A_i)$.
\item Compute posterior $P(A_i\mid E)=\dfrac{P(E\mid A_i)P(A_i)}{P(E)}$.
\item Verify results via complements or contingency-table totals.
\end{enumerate}

\section{Key takeaways}
\begin{itemize}
\item Bayes' theorem is a tool for updating beliefs with evidence.
\item Always check which probabilities are priors vs.\ likelihoods.
\item Low prevalence can produce counterintuitive posterior probabilities even with accurate tests.
\item Use the partition form for multiple competing hypotheses.
\end{itemize}