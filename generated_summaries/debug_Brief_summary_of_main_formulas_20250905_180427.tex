
\documentclass[11pt,a4paper]{article}
\usepackage[utf8]{inputenc}
\usepackage[T1]{fontenc}
\usepackage{lmodern}
\usepackage{microtype}
\usepackage[a4paper,margin=25mm]{geometry}
\usepackage{xcolor}
\definecolor{primary}{HTML}{0B6FA4}
\definecolor{secondary}{HTML}{3A6B8A}
\definecolor{accent}{HTML}{6E7B8C}
\definecolor{lightgray}{HTML}{F3F5F7}
\usepackage{graphicx}
\usepackage{float}
\usepackage{amsmath,amssymb,amsthm}
\usepackage{mathtools}
\usepackage{enumitem}
\usepackage{booktabs}
\usepackage{array}
\usepackage{tcolorbox}
\tcbuselibrary{skins,breakable}
\usepackage{sectsty}
\allsectionsfont{\color{primary}\sffamily}
\usepackage{titlesec}
\titleformat{\section}{\Large\bfseries\sffamily\color{primary}}{\thesection}{1em}{}
\titleformat{\subsection}{\large\bfseries\sffamily\color{secondary}}{\thesubsection}{0.8em}{}
\titleformat{\subsubsection}{\normalsize\bfseries\sffamily\color{accent}}{\thesubsubsection}{0.6em}{}
\usepackage{fancyhdr}
\pagestyle{fancy}
\fancyhf{}
\renewcommand{\headrulewidth}{0.6pt}
\renewcommand{\footrulewidth}{0.4pt}
\renewcommand{\headrule}{\color{lightgray}\hrule height \headrulewidth \vspace{-\headrulewidth}}
\fancyhead[L]{\sffamily\small Course Notes Summary}
\fancyhead[C]{\sffamily\small Focus: Brief summary of main formulas}
\fancyhead[R]{\sffamily\small Source: BayesTheorem.pdf}
\fancyfoot[L]{\sffamily\small Generated: 2025-09-05 18:04:27}
\fancyfoot[C]{}
\fancyfoot[R]{\sffamily\small \thepage}
\usepackage{hyperref}
\hypersetup{colorlinks=true, linkcolor=primary, urlcolor=secondary, citecolor=primary}
\setcounter{secnumdepth}{3}
\setcounter{tocdepth}{2}
\raggedbottom
\usepackage{titling}
\pretitle{\begin{center}\vspace*{1cm}\sffamily}
\posttitle{\par\end{center}\vskip 0.2em}
\preauthor{}\postauthor{}
\predate{}\postdate{}
% Helper macros used by model output
\newcommand{\Prob}{\mathrm{P}}
\newcommand{\given}{\,\mid\,}

\begin{document}


\begin{titlepage}
  \thispagestyle{empty}
  \begin{center}
    \vspace*{1cm}
    \begin{center}
    \includegraphics[width=0.3\textwidth]{16.png}
    \end{center}
    \vspace{1em}
    {\sffamily\Huge\bfseries\color{primary} Course Notes Summary \par}
    \vspace{0.8em}
    {\sffamily\Large\color{secondary} Focus Area: Brief summary of main formulas \par}
    \vspace{1.5em}
    \begin{tcolorbox}[enhanced,width=0.75\textwidth,colback=lightgray,colframe=primary,boxrule=0.8pt,sharp corners,left=6pt,right=6pt,top=6pt,bottom=6pt]\vspace{0.5em}
      \sffamily
      \begin{tabular}{@{}p{0.28\textwidth} p{0.62\textwidth}@{}}
        \textbf{Source: } & BayesTheorem.pdf \\
        \textbf{Generated:} & 2025-09-05 18:04:27 \\
        \textbf{Summary Type:} & Comprehensive \\
      \end{tabular}
    \end{tcolorbox}

    \vspace{1.5em}
    {\small\sffamily Prepared for quick revision and reference\par}
    \vfill

    {\small\sffamily \color{accent} Use this sheet as a step-by-step guide when solving problems.}
    \vspace{1.8cm}
  \end{center}
\end{titlepage}

\section{Definitions}
\begin{itemize}
\item Prior probability: initial probability of an event before new information is observed.
\item Posterior probability: revised probability of an event after new information is observed.
\item Conditional probability: probability of event $B$ given event $A$ has occurred, written $P(B\mid A)$.
\end{itemize}

\section{Main formulas}
\begin{itemize}
\item Conditional probability (basic):
$$
P(B\mid A)=\frac{P(A\text{ and }B)}{P(A)}.
$$
\item Bayes' theorem (two events):
$$
P(B\mid A)=\frac{P(A\mid B)\,P(B)}{P(A)}.
$$
\item Total probability (two-event form; partition $B,B^c$):
$$
P(A)=P(B)P(A\mid B)+P(B^c)P(A\mid B^c),
$$
(use appropriate partitioning; more generally below).
\item Bayes' theorem (generalized, partition $A_1,\dots,A_n$):
$$
P(A_k\mid B)=\frac{P(A_k)\,P(B\mid A_k)}{\displaystyle\sum_{j=1}^n P(A_j)\,P(B\mid A_j)}.
$$
\item Total probability (generalized):
$$
P(B)=\sum_{j=1}^n P(A_j)\,P(B\mid A_j).
$$
\end{itemize}

\section{Intuitive/frequency formulation}
\begin{itemize}
\item Replace probabilities by counts using an assumed total $N$.
\item Compute cell counts:
$$
\text{count}(A_j\text{ and }B)=N\cdot P(A_j)\cdot P(B\mid A_j).
$$
\item Then
$$
P(A_k\mid B)=\frac{\text{count}(A_k\text{ and }B)}{\sum_j \text{count}(A_j\text{ and }B)}.
$$
\item This produces the same numeric result as Bayes' formula and helps avoid substitution errors.
\end{itemize}

\section{Examples}
\subsection{ELT manufacturers (three events)}
Given:
\begin{itemize}
\item $P(A)=0.80$, $P(B)=0.15$, $P(C)=0.05$.
\item $P(D\mid A)=0.04$, $P(D\mid B)=0.06$, $P(D\mid C)=0.09$.
\end{itemize}
Compute $P(A\mid D)$:
$$
P(A\mid D)=\frac{P(A)\,P(D\mid A)}{P(A)P(D\mid A)+P(B)P(D\mid B)+P(C)P(D\mid C)}
$$
$$
=\frac{0.80\cdot 0.04}{0.80\cdot 0.04 + 0.15\cdot 0.06 + 0.05\cdot 0.09}
=\frac{0.032}{0.032 + 0.009 + 0.0045}
=\frac{0.032}{0.0455}\approx 0.703.
$$
Frequency check ($N=10{,}000$): defective counts = $320$ (A), $90$ (B), $45$ (C), total $455$. $P=320/455\approx 0.703$.

\subsection{Orange County cigar example}
Given:
\begin{itemize}
\item $P(\text{male})=0.51$, $P(\text{female})=0.49$.
\item $P(\text{cigar}\mid \text{male})=0.095$, $P(\text{cigar}\mid \text{female})=0.017$.
\end{itemize}
Compute $P(\text{male}\mid \text{cigar})$:
$$
P(\text{cigar})=0.51\cdot 0.095 + 0.49\cdot 0.017 = 0.04845 + 0.00833 = 0.05678,
$$
$$
P(\text{male}\mid \text{cigar})=\frac{0.04845}{0.05678}\approx 0.853.
$$
Frequency check ($N=100{,}000$): male smokers $=4845$, female smokers $=833$, total smokers $=5678$. $P=4845/5678\approx 0.853$.

\section{Application steps (how to use Bayes' theorem)}
\begin{enumerate}
\item Identify the hypothesis events $A_1,\dots,A_n$ that form a partition of the sample space.
\item Determine prior probabilities $P(A_j)$.
\item Determine likelihoods $P(B\mid A_j)$ for the observed evidence $B$.
\item Compute $P(B)$ via total probability: $P(B)=\sum_j P(A_j)\,P(B\mid A_j)$.
\item Compute posterior: $P(A_k\mid B)=\dfrac{P(A_k)\,P(B\mid A_k)}{P(B)}$.
\end{enumerate}

\section{Important relationships and connections}
\begin{itemize}
\item Posterior \propto Prior $\times$ Likelihood:
$$
P(A_k\mid B)\propto P(A_k)\,P(B\mid A_k).
$$
\item Total probability provides the normalizing constant (denominator) for Bayes' theorem.
\item The frequency-table (count) method is algebraically equivalent to the formula method but often easier to implement and check by inspection.
\item Bayes' theorem is sequential: posteriors from one update can serve as priors for the next update when new evidence arrives.
\end{itemize}

\section{Notes and pitfalls}
\begin{itemize}
\item Ensure the $A_j$ are a complete partition and $P(B)>0$.
\item Distinguish carefully between $P(B\mid A)$ and $P(A\mid B)$; they are generally not equal.
\item Use the frequency/table method to reduce algebraic substitution errors.
\item Round only at the final step to avoid cumulative rounding error.
\end{itemize}

\section{Quick reference (compact formulas)}
\begin{itemize}
\item $P(B\mid A)=\dfrac{P(A\text{ and }B)}{P(A)}$.
\item $P(B\mid A)=\dfrac{P(A\mid B)\,P(B)}{P(A)}$.
\item $P(B)=\sum_j P(A_j)\,P(B\mid A_j)$.
\item $P(A_k\mid B)=\dfrac{P(A_k)\,P(B\mid A_k)}{\displaystyle\sum_j P(A_j)\,P(B\mid A_j)}$.
\end{itemize}

\end{document}
