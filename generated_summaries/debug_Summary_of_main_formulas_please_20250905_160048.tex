
\documentclass[11pt,a4paper]{article}
\usepackage[utf8]{inputenc}
\usepackage[T1]{fontenc}
\usepackage{lmodern}
\usepackage{microtype}
\usepackage[a4paper,margin=25mm]{geometry}
\usepackage{xcolor}
\definecolor{primary}{HTML}{0B6FA4}
\definecolor{secondary}{HTML}{3A6B8A}
\definecolor{accent}{HTML}{6E7B8C}
\definecolor{lightgray}{HTML}{F3F5F7}
\usepackage{graphicx}
\usepackage{float}
\usepackage{amsmath,amssymb,amsthm}
\usepackage{mathtools}
\usepackage{enumitem}
\usepackage{booktabs}
\usepackage{array}
\usepackage{tcolorbox}
\tcbuselibrary{skins,breakable}
\usepackage{sectsty}
\allsectionsfont{\color{primary}\sffamily}
\usepackage{titlesec}
\titleformat{\section}{\Large\bfseries\sffamily\color{primary}}{\thesection}{1em}{}
\titleformat{\subsection}{\large\bfseries\sffamily\color{secondary}}{\thesubsection}{0.8em}{}
\titleformat{\subsubsection}{\normalsize\bfseries\sffamily\color{accent}}{\thesubsubsection}{0.6em}{}
\usepackage{fancyhdr}
\pagestyle{fancy}
\fancyhf{}
\renewcommand{\headrulewidth}{0.6pt}
\renewcommand{\footrulewidth}{0.4pt}
\renewcommand{\headrule}{\color{lightgray}\hrule height \headrulewidth \vspace{-\headrulewidth}}
\fancyhead[L]{\sffamily\small Course Notes Summary}
\fancyhead[C]{\sffamily\small Focus: Summary of main formulas please}
\fancyhead[R]{\sffamily\small Source: BayesTheorem.pdf}
\fancyfoot[L]{\sffamily\small Generated: 2025-09-05 16:00:48}
\fancyfoot[C]{}
\fancyfoot[R]{\sffamily\small \thepage}
\usepackage{hyperref}
\hypersetup{colorlinks=true, linkcolor=primary, urlcolor=secondary, citecolor=primary}
\setcounter{secnumdepth}{3}
\setcounter{tocdepth}{2}
\raggedbottom
\usepackage{titling}
\pretitle{\begin{center}\vspace*{1cm}\sffamily}
\posttitle{\par\end{center}\vskip 0.2em}
\preauthor{}\postauthor{}
\predate{}\postdate{}
% Helper macros used by model output
\newcommand{\Prob}{\mathrm{P}}
\newcommand{\given}{\,\mid\,}

\begin{document}


\begin{titlepage}
  \thispagestyle{empty}
  \begin{center}
    \vspace*{1cm}
    \includegraphics[width=0.3\textwidth]{16.png}
    \vspace{1em}
    {\sffamily\Huge\bfseries\color{primary} Course Notes Summary \par}
    \vspace{0.8em}
    {\sffamily\Large\color{secondary} Focus Area: Summary of main formulas please \par}
    \vspace{1.5em}
    \begin{tcolorbox}[enhanced,width=0.75\textwidth,colback=lightgray,colframe=primary,boxrule=0.8pt,sharp corners,left=6pt,right=6pt,top=6pt,bottom=6pt]\vspace{0.5em}
      \sffamily
      \begin{tabular}{@{}p{0.28\textwidth} p{0.62\textwidth}@{}}
        \textbf{Source: } & BayesTheorem.pdf \\
        \textbf{Generated:} & 2025-09-05 16:00:48 \\
        \textbf{Summary Type:} & Comprehensive \\
      \end{tabular}
    \end{tcolorbox}

    \vspace{1.5em}
    {\small\sffamily Prepared for quick revision and reference\par}
    \vfill

    {\small\sffamily \color{accent} Use this sheet as a step-by-step guide when solving problems.}
    \vspace{1.8cm}
  \end{center}
\end{titlepage}

\section{Fundamental definitions}
\begin{itemize}
\item Prior probability: initial probability before new information.
\item Posterior probability: probability revised after new information.
\item Conditional probability: probability of $B$ given $A$ has occurred.
\item Partition: a set of mutually exclusive, exhaustive events $A_1,\ldots,A_n$.
\end{itemize}

\section{Core formulas}

\subsection{Conditional probability}
\begin{itemize}
\item $P(B\mid A)=\dfrac{P(A\cap B)}{P(A)}$, for $P(A)>0$.
\item Equivalent product form: $P(A\cap B)=P(A)\,P(B\mid A)=P(B)\,P(A\mid B)$.
\end{itemize}

\subsection{Two-event Bayes' theorem}
\begin{itemize}
\item $P(A\mid B)=\dfrac{P(B\mid A)\,P(A)}{P(B)}$, provided $P(B)>0$.
\item Law of total probability for denominator (two events $A$ and $A^c$):
\[
P(B)=P(B\mid A)\,P(A)+P(B\mid A^c)\,P(A^c).
\]
\end{itemize}

\subsection{Generalized (n-event) Bayes' theorem}
\begin{itemize}
\item Law of total probability (partition $A_1,\ldots,A_n$):
\[
P(B)=\sum_{i=1}^n P(B\mid A_i)\,P(A_i).
\]
\item General Bayes formula for $A_j$ in partition:
\[
P(A_j\mid B)=\frac{P(B\mid A_j)\,P(A_j)}{\displaystyle\sum_{i=1}^n P(B\mid A_i)\,P(A_i)}.
\]
\end{itemize}

\subsection{Intuitive / frequency (tree-table) method}
\begin{itemize}
\item Choose a convenient total $N$ (e.g., $N=100{,}000$).
\item Compute cell counts: $\text{count}(A_i\cap B)=N\,P(A_i)\,P(B\mid A_i)$.
\item Use counts to get posterior:
\[
P(A_j\mid B)=\frac{\text{count}(A_j\cap B)}{\sum_i \text{count}(A_i\cap B)}.
\]
\end{itemize}

\section{Short worked examples}

\subsection{Cigar-smoking example (two-category)}
\begin{itemize}
\item Given: $P(\text{male})=0.51$, $P(\text{cigar}\mid\text{male})=0.095$, $P(\text{cigar}\mid\text{female})=0.017$.
\item Denominator:
\[
P(\text{cigar})=0.095\times 0.51 + 0.017\times 0.49 = 0.04845 + 0.00833 = 0.05678.
\]
\item Posterior:
\[
P(\text{male}\mid\text{cigar})=\frac{0.095\times 0.51}{0.05678}\approx\frac{0.04845}{0.05678}\approx 0.854.
\]
\item Frequency check ($N=100{,}000$): males smoking $=100{,}000\times 0.51\times 0.095=4{,}845$; females smoking $=100{,}000\times 0.49\times 0.017=833$; total $=5{,}678$; ratio $=4{,}845/5{,}678\approx 0.854$.
\end{itemize}

\subsection{ELT manufacturers (three-category)}
\begin{itemize}
\item Given: $P(A)=0.80$, $P(B)=0.15$, $P(C)=0.05$.
\item Defective rates: $P(D\mid A)=0.04$, $P(D\mid B)=0.06$, $P(D\mid C)=0.09$.
\item Numerator for $A$: $0.80\times 0.04 = 0.032$.
\item Denominator:
\[
0.032 + 0.15\times 0.06 + 0.05\times 0.09 = 0.032 + 0.009 + 0.0045 = 0.0455.
\]
\item Posterior:
\[
P(A\mid D)=\frac{0.032}{0.0455}\approx 0.703.
\]
\item Frequency check ($N=10{,}000$): defective from $A=10{,}000\times 0.80\times 0.04=320$; total defective $=455$; ratio $=320/455\approx 0.703$.
\end{itemize}

\section{Key relationships and connections}
\begin{itemize}
\item Bayes updates prior to posterior using likelihood $P(B\mid A_i)$.
\item Denominator always equals weighted sum of likelihoods across partition (law of total probability).
\item Product rule links joint and conditional probabilities: $P(A\cap B)=P(A)\,P(B\mid A)$.
\item Using frequencies reduces algebra mistakes and improves intuition.
\end{itemize}

\section{Quick application checklist (ordered steps)}
\begin{enumerate}
\item Identify hypotheses/partitions $A_1,\ldots,A_n$ and the observed event $B$.
\item Collect priors $P(A_i)$ and likelihoods $P(B\mid A_i)$.
\item Compute denominator: $\sum_i P(B\mid A_i)\,P(A_i)$.
\item Compute posterior for desired $A_j$: $P(A_j\mid B)=\dfrac{P(B\mid A_j)\,P(A_j)}{\text{denominator}}$.
\item Optionally verify with a frequency table using a chosen $N$.
\end{enumerate}

\section{Common pitfalls and notes}
\begin{itemize}
\item Ensure events $A_i$ form a partition (mutually exclusive and exhaustive).
\item Check that $P(B)$ (denominator) $>0$ before dividing.
\item Distinguish $P(B\mid A)$ (likelihood) from $P(A\mid B)$ (posterior).
\item Use the frequency method to avoid algebraic substitution errors.
\end{itemize}

\section{Compact formula summary (for quick revision)}
\begin{itemize}
\item Conditional: $P(B\mid A)=\dfrac{P(A\cap B)}{P(A)}$.
\item Product: $P(A\cap B)=P(A)\,P(B\mid A)$.
\item Bayes (two-event): $P(A\mid B)=\dfrac{P(B\mid A)\,P(A)}{P(B)}$.
\item Total prob. (partition): $P(B)=\sum_i P(B\mid A_i)\,P(A_i)$.
\item Bayes (general): $P(A_j\mid B)=\dfrac{P(B\mid A_j)\,P(A_j)}{\displaystyle\sum_i P(B\mid A_i)\,P(A_i)}$.
\end{itemize}

\end{document}
